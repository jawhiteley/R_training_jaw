\setbeamertemplate{itemize item}[circle]
\setbeamertemplate{enumerate item}[circle]
%\useoutertheme{miniframes}    % for progress bars along the top of each frame
%\usepackage{layout}    % add `\layout` to the document to show page layout and parameter values
\usepackage{tikz}
\usetikzlibrary{arrows.meta,calc,tikzmark,fit, positioning}
% https://tex.stackexchange.com/questions/136143/tikz-animated-figure-in-beamer
% This won't hide parts of a drawing that have an opacity set in the picture or manually (it overrides the opacity here).
% see also the `overlay-beamer-styles` tikz library: https://tex.stackexchange.com/questions/639922/using-tikz-with-pause-command-in-rmarkdown-or-quarto-presentation
\tikzset{
  invisible/.style={opacity=0},
  visible on/.style={alt={#1{}{invisible}}},
  alt/.code args={<#1>#2#3}{%
    \alt<#1>{\pgfkeysalso{#2}}{\pgfkeysalso{#3}} % \pgfkeysalso doesn't change the path
  },
}
\AtBeginEnvironment{verbatim}{\small}    % reduce font size in R output
% https://stackoverflow.com/questions/35734525/reduce-space-between-code-chunks-and-code-output-in-rmarkdown-beamer-presentatio
\makeatletter
  \preto{\@verbatim}{\topsep=0pt \partopsep=0pt \itemsep=0pt}
\makeatother
% Set some lengths explicitly, for reproducibility across environments with different defaults
% Using fixed values (without rubber lengths) allows for more predictability in sizes,
%  which is useful for graphical overlays, though it may not be as flexible.
% See p. 145 of "The LaTeX Companion" (2e) for a diagram and explanation of some of these parameters.
\setlength{\partopsep}{0pt}
\setlength{\topsep}{0pt}
% My first instinct is to set \parsep to 0 (so paragraphs in the same item are closer together)
%  and set \itemsep to a rubber value; but the default is the opposite!
%  So all paragraphs are the same distance apart, regardless of item number...
% The \tightlist command sets \itemsep and \parskip to 0, but does not change \parsep.
\setlength{\parsep}{6pt plus2pt minus2pt}
\setlength{\itemsep}{0pt}
% For the most part, I don't actually want any space around the Shaded environments,
%  but it's the easiest way I found to reliably reserve space above the code chunks in columns underneath code output (verbatim).
\setlength{\OuterFrameSep}{4pt}  % = \topsep by default in `framed` package (but this is only applied at the start of the environment: the default value of \OuterFrameSep is otherwise \maxdimen!)
\newlength\ShadedFrameSep 
\ifdim\OuterFrameSep=\maxdimen 
  \ShadedFrameSep=\topsep \else 
  \ShadedFrameSep=\OuterFrameSep 
\fi
%% Custom colours
\definecolor{darkred}{rgb}{0.8,0,0}
\definecolor{darkblue}{rgb}{0,0,0.6}
%% Custom macros
\newcommand{\ctop}{\vspace{\ShadedFrameSep}}  % for aligning text with the top of a code block in a table (0.2\baselinesep seemed like a good approximation, depending on other parametes; '\topsep+\parskip+\partopsep' seemed logical, but might be too much, maybe because it loses the rubber values)
\newcommand{\chup}{\vspace{-\ShadedFrameSep}}  % move code chunk up, to gobble whitespace above?
%\newcommand{\Rlogo}{\includegraphics[height=1em]{`r file.path(R.home("doc"), "html", "logo.jpg")`}}  %% needs to go through `knit()` to be valid.
\newcommand{\R}{\texttt{R}}
\newcommand{\highlight}[1]{\StringTok{#1}}
\newcommand{\important}[1]{\textcolor{darkred}{#1}}
\newcommand{\fade}[1]{\textcolor[rgb]{0.66,0.66,0.66}{#1}}
\newcommand{\annote}[1]{{\footnotesize #1}}
\newcommand{\name}[1]{\VariableTok{\texttt{#1}}}

