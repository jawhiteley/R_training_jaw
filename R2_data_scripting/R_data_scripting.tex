% Options for packages loaded elsewhere
\PassOptionsToPackage{unicode}{hyperref}
\PassOptionsToPackage{hyphens}{url}
\PassOptionsToPackage{dvipsnames,svgnames,x11names}{xcolor}
%
\documentclass[
  11pt,
  ignorenonframetext,
]{beamer}
\usepackage{pgfpages}
\setbeamertemplate{caption}[numbered]
\setbeamertemplate{caption label separator}{: }
\setbeamercolor{caption name}{fg=normal text.fg}
\beamertemplatenavigationsymbolsempty
% Prevent slide breaks in the middle of a paragraph
\widowpenalties 1 10000
\raggedbottom
\setbeamertemplate{part page}{
  \centering
  \begin{beamercolorbox}[sep=16pt,center]{part title}
    \usebeamerfont{part title}\insertpart\par
  \end{beamercolorbox}
}
\setbeamertemplate{section page}{
  \centering
  \begin{beamercolorbox}[sep=12pt,center]{part title}
    \usebeamerfont{section title}\insertsection\par
  \end{beamercolorbox}
}
\setbeamertemplate{subsection page}{
  \centering
  \begin{beamercolorbox}[sep=8pt,center]{part title}
    \usebeamerfont{subsection title}\insertsubsection\par
  \end{beamercolorbox}
}
\AtBeginPart{
  \frame{\partpage}
}
\AtBeginSection{
  \ifbibliography
  \else
    \frame{\sectionpage}
  \fi
}
\AtBeginSubsection{
  \frame{\subsectionpage}
}
\usepackage{amsmath,amssymb}
\usepackage{iftex}
\ifPDFTeX
  \usepackage[T1]{fontenc}
  \usepackage[utf8]{inputenc}
  \usepackage{textcomp} % provide euro and other symbols
\else % if luatex or xetex
  \usepackage{unicode-math} % this also loads fontspec
  \defaultfontfeatures{Scale=MatchLowercase}
  \defaultfontfeatures[\rmfamily]{Ligatures=TeX,Scale=1}
\fi
\usepackage{lmodern}
\usetheme[]{Boadilla}
\ifPDFTeX\else
  % xetex/luatex font selection
\fi
% Use upquote if available, for straight quotes in verbatim environments
\IfFileExists{upquote.sty}{\usepackage{upquote}}{}
\IfFileExists{microtype.sty}{% use microtype if available
  \usepackage[]{microtype}
  \UseMicrotypeSet[protrusion]{basicmath} % disable protrusion for tt fonts
}{}
\makeatletter
\@ifundefined{KOMAClassName}{% if non-KOMA class
  \IfFileExists{parskip.sty}{%
    \usepackage{parskip}
  }{% else
    \setlength{\parindent}{0pt}
    \setlength{\parskip}{6pt plus 2pt minus 1pt}}
}{% if KOMA class
  \KOMAoptions{parskip=half}}
\makeatother
\usepackage{xcolor}
\newif\ifbibliography
\usepackage{color}
\usepackage{fancyvrb}
\newcommand{\VerbBar}{|}
\newcommand{\VERB}{\Verb[commandchars=\\\{\}]}
\DefineVerbatimEnvironment{Highlighting}{Verbatim}{commandchars=\\\{\}}
% Add ',fontsize=\small' for more characters per line
\usepackage{framed}
\definecolor{shadecolor}{RGB}{242,242,248}
\newenvironment{Shaded}{\begin{snugshade}}{\end{snugshade}}
\newcommand{\AlertTok}[1]{\textcolor[rgb]{0.94,0.16,0.16}{#1}}
\newcommand{\AnnotationTok}[1]{\textcolor[rgb]{0.56,0.35,0.01}{\textbf{\textit{#1}}}}
\newcommand{\AttributeTok}[1]{\textcolor[rgb]{0.13,0.29,0.53}{#1}}
\newcommand{\BaseNTok}[1]{\textcolor[rgb]{0.00,0.00,0.81}{#1}}
\newcommand{\BuiltInTok}[1]{#1}
\newcommand{\CharTok}[1]{\textcolor[rgb]{0.31,0.60,0.02}{#1}}
\newcommand{\CommentTok}[1]{\textcolor[rgb]{0.56,0.35,0.01}{\textit{#1}}}
\newcommand{\CommentVarTok}[1]{\textcolor[rgb]{0.56,0.35,0.01}{\textbf{\textit{#1}}}}
\newcommand{\ConstantTok}[1]{\textcolor[rgb]{0.56,0.35,0.01}{#1}}
\newcommand{\ControlFlowTok}[1]{\textcolor[rgb]{0.13,0.29,0.53}{\textbf{#1}}}
\newcommand{\DataTypeTok}[1]{\textcolor[rgb]{0.13,0.29,0.53}{#1}}
\newcommand{\DecValTok}[1]{\textcolor[rgb]{0.00,0.00,0.81}{#1}}
\newcommand{\DocumentationTok}[1]{\textcolor[rgb]{0.56,0.35,0.01}{\textbf{\textit{#1}}}}
\newcommand{\ErrorTok}[1]{\textcolor[rgb]{0.64,0.00,0.00}{\textbf{#1}}}
\newcommand{\ExtensionTok}[1]{#1}
\newcommand{\FloatTok}[1]{\textcolor[rgb]{0.00,0.00,0.81}{#1}}
\newcommand{\FunctionTok}[1]{\textcolor[rgb]{0.13,0.29,0.53}{\textbf{#1}}}
\newcommand{\ImportTok}[1]{#1}
\newcommand{\InformationTok}[1]{\textcolor[rgb]{0.56,0.35,0.01}{\textbf{\textit{#1}}}}
\newcommand{\KeywordTok}[1]{\textcolor[rgb]{0.13,0.29,0.53}{\textbf{#1}}}
\newcommand{\NormalTok}[1]{#1}
\newcommand{\OperatorTok}[1]{\textcolor[rgb]{0.81,0.36,0.00}{\textbf{#1}}}
\newcommand{\OtherTok}[1]{\textcolor[rgb]{0.56,0.35,0.01}{#1}}
\newcommand{\PreprocessorTok}[1]{\textcolor[rgb]{0.56,0.35,0.01}{\textit{#1}}}
\newcommand{\RegionMarkerTok}[1]{#1}
\newcommand{\SpecialCharTok}[1]{\textcolor[rgb]{0.81,0.36,0.00}{\textbf{#1}}}
\newcommand{\SpecialStringTok}[1]{\textcolor[rgb]{0.31,0.60,0.02}{#1}}
\newcommand{\StringTok}[1]{\textcolor[rgb]{0.31,0.60,0.02}{#1}}
\newcommand{\VariableTok}[1]{\textcolor[rgb]{0.00,0.00,0.00}{#1}}
\newcommand{\VerbatimStringTok}[1]{\textcolor[rgb]{0.31,0.60,0.02}{#1}}
\newcommand{\WarningTok}[1]{\textcolor[rgb]{0.56,0.35,0.01}{\textbf{\textit{#1}}}}
\usepackage{graphicx}
\makeatletter
\def\maxwidth{\ifdim\Gin@nat@width>\linewidth\linewidth\else\Gin@nat@width\fi}
\def\maxheight{\ifdim\Gin@nat@height>\textheight\textheight\else\Gin@nat@height\fi}
\makeatother
% Scale images if necessary, so that they will not overflow the page
% margins by default, and it is still possible to overwrite the defaults
% using explicit options in \includegraphics[width, height, ...]{}
\setkeys{Gin}{width=\maxwidth,height=\maxheight,keepaspectratio}
% Set default figure placement to htbp
\makeatletter
\def\fps@figure{htbp}
\makeatother
\setlength{\emergencystretch}{3em} % prevent overfull lines
\providecommand{\tightlist}{%
  \setlength{\itemsep}{0pt}\setlength{\parskip}{0pt}}
\setcounter{secnumdepth}{-\maxdimen} % remove section numbering
%% Custom LaTeX code to include in Beamer presentations from RMarkdown
%% Jonathan Whiteley, 2023
%% To be included in RMarkdown files:
%%    includes:
%%      in_header: "preamble-include.tex"
\setbeamertemplate{itemize item}[circle]
\setbeamertemplate{enumerate item}[circle]
%\useoutertheme{miniframes}    % for progress bars along the top of each frame
%\usepackage{layout}    % add `\layout` to the document to show page layout and parameter values
\usepackage{tikz}
\usetikzlibrary{arrows.meta,calc,tikzmark,fit, positioning}
% https://tex.stackexchange.com/questions/136143/tikz-animated-figure-in-beamer
% This won't hide parts of a drawing that have an opacity set in the picture or manually (it overrides the opacity here).
% see also the `overlay-beamer-styles` tikz library: https://tex.stackexchange.com/questions/639922/using-tikz-with-pause-command-in-rmarkdown-or-quarto-presentation
\tikzset{
  invisible/.style={opacity=0},
  visible on/.style={alt={#1{}{invisible}}},
  alt/.code args={<#1>#2#3}{%
    \alt<#1>{\pgfkeysalso{#2}}{\pgfkeysalso{#3}} % \pgfkeysalso doesn't change the path
  },
}
\AtBeginEnvironment{verbatim}{\small}    % reduce font size in R output
% https://stackoverflow.com/questions/35734525/reduce-space-between-code-chunks-and-code-output-in-rmarkdown-beamer-presentatio
\makeatletter
  \preto{\@verbatim}{\topsep=0pt \partopsep=0pt \itemsep=0pt}
\makeatother
% Set some lengths explicitly, for reproducibility across environments with different defaults
% Using fixed values (without rubber lengths) allows for more predictability in sizes,
%  which is useful for graphical overlays, though it may not be as flexible.
% See p. 145 of "The LaTeX Companion" (2e) for a diagram and explanation of some of these parameters.
\setlength{\partopsep}{0pt}
\setlength{\topsep}{0pt}
% My first instinct is to set \parsep to 0 (so paragraphs in the same item are closer together)
%  and set \itemsep to a rubber value; but the default is the opposite!
%  So all paragraphs are the same distance apart, regardless of item number...
% The \tightlist command sets \itemsep and \parskip to 0, but does not change \parsep.
\setlength{\parsep}{6pt plus2pt minus2pt}
\setlength{\itemsep}{0pt}
% For the most part, I don't actually want any space around the Shaded environments,
%  but it's the easiest way I found to reliably reserve space above the code chunks in columns underneath code output (verbatim).
\makeatletter
\@ifpackageloaded{framed}{
  \setlength{\OuterFrameSep}{4pt}  % = \topsep by default in `framed` package (but this is only applied at the start of the environment: the default value of \OuterFrameSep is otherwise \maxdimen!)
  \newlength\ShadedFrameSep 
  \ifdim\OuterFrameSep=\maxdimen 
    \ShadedFrameSep=\topsep \else 
    \ShadedFrameSep=\OuterFrameSep 
  \fi
}{}
\makeatother
%% Custom colours
\definecolor{darkred}{rgb}{0.8,0,0}
\definecolor{darkblue}{rgb}{0,0,0.6}
%% Custom macros
\newcommand{\ctop}{\vspace{\ShadedFrameSep}}  % for aligning text with the top of a code block in a table (0.2\baselinesep seemed like a good approximation, depending on other parametes; '\topsep+\parskip+\partopsep' seemed logical, but might be too much, maybe because it loses the rubber values)
\newcommand{\chup}{\vspace{-\ShadedFrameSep}}  % move code chunk up, to gobble whitespace above?
%\newcommand{\Rlogo}{\includegraphics[height=1em]{`r file.path(R.home("doc"), "html", "logo.jpg")`}}  %% needs to go through `knit()` to be valid.
\newcommand{\R}{\texttt{R}}
\newcommand{\highlight}[1]{\StringTok{#1}}
\newcommand{\important}[1]{\textcolor{darkred}{#1}}
\newcommand{\fade}[1]{\textcolor[rgb]{0.66,0.66,0.66}{#1}}
\newcommand{\annote}[1]{{\footnotesize #1}}
\newcommand{\name}[1]{\VariableTok{\texttt{#1}}}

\ifLuaTeX
  \usepackage{selnolig}  % disable illegal ligatures
\fi
\IfFileExists{bookmark.sty}{\usepackage{bookmark}}{\usepackage{hyperref}}
\IfFileExists{xurl.sty}{\usepackage{xurl}}{} % add URL line breaks if available
\urlstyle{same}
\hypersetup{
  pdftitle={A Short Introduction to Working With Data in R},
  pdfauthor={Jonathan Whiteley},
  colorlinks=true,
  linkcolor={darkblue},
  filecolor={Maroon},
  citecolor={Blue},
  urlcolor={darkblue},
  pdfcreator={LaTeX via pandoc}}

\title{A Short Introduction to Working With Data in R}
\author{Jonathan Whiteley}
\date{2023-08-24}

\begin{document}
\frame{\titlepage}

\begin{frame}[fragile]{Prerequisites}
\protect\hypertarget{prerequisites}{}
\begin{itemize}
\item
  Access to a copy of the
  \href{https://www.r-project.org}{\includegraphics[width=\textwidth,height=1em]{/Library/Frameworks/R.framework/Resources/doc/html/logo.jpg}}
  software

  \begin{itemize}
  \item
    i.e., a ``binary executable''
  \item
    Go to \href{https://www.r-project.org}{\emph{www.r-project.org}} to
    get a copy,\\
    or ask your system administrator.
  \end{itemize}
\item
  Tidyverse packages installed on the same system as \(\R\)

  \begin{itemize}
  \item
    Please run this command in \(\R\) \emph{before} the workshop:

\begin{Shaded}
\begin{Highlighting}[]
\FunctionTok{install.packages}\NormalTok{(}\StringTok{"tidyverse"}\NormalTok{)}
\end{Highlighting}
\end{Shaded}
  \end{itemize}
\item
  Knowledge of common mathematical operations: arithmetic, logarithms,
  etc.
\item
  Knowledge of basic \(\R\) concepts, such as \emph{variables},
  \emph{objects}, \emph{operators}, \emph{functions}, \emph{packages},
  etc.

  \begin{itemize}
  \tightlist
  \item
    This is covered in the first workshop: ``A Gentle Introduction to
    R''
  \end{itemize}
\end{itemize}
\end{frame}

\begin{frame}{Learning Objectives}
\protect\hypertarget{learning-objectives}{}
\begin{itemize}
\tightlist
\item
  Load tabular data into R
\item
  Explore data to check that it was loaded correctly
\item
  Export data from R to external files
\item
  Data frames
\item
  Clean data

  \begin{itemize}
  \tightlist
  \item
    Add \& change columns
  \item
    Edit values systematically
  \item
    Change data types
  \end{itemize}
\item
  Tidy data

  \begin{itemize}
  \tightlist
  \item
    Change the \emph{shape} of a data frame
  \end{itemize}
\item
  Re-use code, reproducible results, automated reports

  \begin{itemize}
  \tightlist
  \item
    Scripts
  \item
    R Markdown, R Notebooks
  \end{itemize}
\end{itemize}
\end{frame}

\hypertarget{welcome}{%
\section{Welcome}\label{welcome}}

\begin{frame}{Pop Quiz}
\protect\hypertarget{pop-quiz}{}
\annote{\fade{We will review these \textit{at the end}, so you can see how much you have learned.}}

\begin{itemize}
\tightlist
\item
\end{itemize}

~

\begin{block}{Answer in the chat:}
\protect\hypertarget{answer-in-the-chat}{}
What is your favourite emoji? Why do you like to use it so much?
\end{block}
\end{frame}

\begin{frame}{Introductions}
\protect\hypertarget{introductions}{}
\begin{itemize}
\tightlist
\item
  Name
\item
  Pronouns
\item
  Job title, role
\item
  \emph{optional}: a favourite childhood treat or candy?
\end{itemize}

\begin{itemize}
\tightlist
\item
  What are you hoping to learn most in today's workshop?
\end{itemize}
\end{frame}

\hypertarget{loading-data-into-r}{%
\section{\texorpdfstring{Loading data into
\(\R\)}{Loading data into \textbackslash R}}\label{loading-data-into-r}}

\hypertarget{exploring-your-data}{%
\section{Exploring your data}\label{exploring-your-data}}

\begin{frame}{Data frames}
\protect\hypertarget{data-frames}{}
\end{frame}

\hypertarget{saving-data-outside-r}{%
\section{\texorpdfstring{Saving data outside
\(\R\)}{Saving data outside \textbackslash R}}\label{saving-data-outside-r}}

\hypertarget{re-using-your-code-scripts-and-other-files}{%
\section{Re-using your code: scripts and other
files}\label{re-using-your-code-scripts-and-other-files}}

\hypertarget{the-tidyverse-collection-of-packages}{%
\section{\texorpdfstring{The \texttt{tidyverse} collection of
packages}{The tidyverse collection of packages}}\label{the-tidyverse-collection-of-packages}}

\hypertarget{clean-data}{%
\section{Clean data}\label{clean-data}}

\hypertarget{tidy-data}{%
\section{Tidy data}\label{tidy-data}}

\hypertarget{review}{%
\section{Review}\label{review}}

\begin{frame}{Exercise}
\protect\hypertarget{exercise}{}
\end{frame}

\begin{frame}[fragile]{\protect\hyperlink{pop-quiz}{Quiz} Review}
\protect\hypertarget{quiz-review}{}
\note{\texttt{\#\#\ Answers\ (for\ discussion)}}
\end{frame}

\hypertarget{backmatter}{%
\section{Backmatter}\label{backmatter}}

\begin{frame}{References}
\protect\hypertarget{references}{}
\note{\url{https://cran.rapporter.net/doc/contrib/de_Jonge+van_der_Loo-Introduction_to_data_cleaning_with_R.pdf}}
\end{frame}

\end{document}
