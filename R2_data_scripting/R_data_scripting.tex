% Options for packages loaded elsewhere
\PassOptionsToPackage{unicode}{hyperref}
\PassOptionsToPackage{hyphens}{url}
\PassOptionsToPackage{dvipsnames,svgnames,x11names}{xcolor}
%
\documentclass[
  11pt,
  ignorenonframetext,
]{beamer}
\usepackage{pgfpages}
\setbeamertemplate{caption}[numbered]
\setbeamertemplate{caption label separator}{: }
\setbeamercolor{caption name}{fg=normal text.fg}
\beamertemplatenavigationsymbolsempty
% Prevent slide breaks in the middle of a paragraph
\widowpenalties 1 10000
\raggedbottom
\setbeamertemplate{part page}{
  \centering
  \begin{beamercolorbox}[sep=16pt,center]{part title}
    \usebeamerfont{part title}\insertpart\par
  \end{beamercolorbox}
}
\setbeamertemplate{section page}{
  \centering
  \begin{beamercolorbox}[sep=12pt,center]{part title}
    \usebeamerfont{section title}\insertsection\par
  \end{beamercolorbox}
}
\setbeamertemplate{subsection page}{
  \centering
  \begin{beamercolorbox}[sep=8pt,center]{part title}
    \usebeamerfont{subsection title}\insertsubsection\par
  \end{beamercolorbox}
}
\AtBeginPart{
  \frame{\partpage}
}
\AtBeginSection{
  \ifbibliography
  \else
    \frame{\sectionpage}
  \fi
}
\AtBeginSubsection{
  \frame{\subsectionpage}
}
\usepackage{amsmath,amssymb}
\usepackage{iftex}
\ifPDFTeX
  \usepackage[T1]{fontenc}
  \usepackage[utf8]{inputenc}
  \usepackage{textcomp} % provide euro and other symbols
\else % if luatex or xetex
  \usepackage{unicode-math} % this also loads fontspec
  \defaultfontfeatures{Scale=MatchLowercase}
  \defaultfontfeatures[\rmfamily]{Ligatures=TeX,Scale=1}
\fi
\usepackage{lmodern}
\usetheme[]{Boadilla}
\ifPDFTeX\else
  % xetex/luatex font selection
\fi
% Use upquote if available, for straight quotes in verbatim environments
\IfFileExists{upquote.sty}{\usepackage{upquote}}{}
\IfFileExists{microtype.sty}{% use microtype if available
  \usepackage[]{microtype}
  \UseMicrotypeSet[protrusion]{basicmath} % disable protrusion for tt fonts
}{}
\makeatletter
\@ifundefined{KOMAClassName}{% if non-KOMA class
  \IfFileExists{parskip.sty}{%
    \usepackage{parskip}
  }{% else
    \setlength{\parindent}{0pt}
    \setlength{\parskip}{6pt plus 2pt minus 1pt}}
}{% if KOMA class
  \KOMAoptions{parskip=half}}
\makeatother
\usepackage{xcolor}
\newif\ifbibliography
\usepackage{color}
\usepackage{fancyvrb}
\newcommand{\VerbBar}{|}
\newcommand{\VERB}{\Verb[commandchars=\\\{\}]}
\DefineVerbatimEnvironment{Highlighting}{Verbatim}{commandchars=\\\{\}}
% Add ',fontsize=\small' for more characters per line
\usepackage{framed}
\definecolor{shadecolor}{RGB}{242,242,248}
\newenvironment{Shaded}{\begin{snugshade}}{\end{snugshade}}
\newcommand{\AlertTok}[1]{\textcolor[rgb]{0.94,0.16,0.16}{#1}}
\newcommand{\AnnotationTok}[1]{\textcolor[rgb]{0.56,0.35,0.01}{\textbf{\textit{#1}}}}
\newcommand{\AttributeTok}[1]{\textcolor[rgb]{0.13,0.29,0.53}{#1}}
\newcommand{\BaseNTok}[1]{\textcolor[rgb]{0.00,0.00,0.81}{#1}}
\newcommand{\BuiltInTok}[1]{#1}
\newcommand{\CharTok}[1]{\textcolor[rgb]{0.31,0.60,0.02}{#1}}
\newcommand{\CommentTok}[1]{\textcolor[rgb]{0.56,0.35,0.01}{\textit{#1}}}
\newcommand{\CommentVarTok}[1]{\textcolor[rgb]{0.56,0.35,0.01}{\textbf{\textit{#1}}}}
\newcommand{\ConstantTok}[1]{\textcolor[rgb]{0.56,0.35,0.01}{#1}}
\newcommand{\ControlFlowTok}[1]{\textcolor[rgb]{0.13,0.29,0.53}{\textbf{#1}}}
\newcommand{\DataTypeTok}[1]{\textcolor[rgb]{0.13,0.29,0.53}{#1}}
\newcommand{\DecValTok}[1]{\textcolor[rgb]{0.00,0.00,0.81}{#1}}
\newcommand{\DocumentationTok}[1]{\textcolor[rgb]{0.56,0.35,0.01}{\textbf{\textit{#1}}}}
\newcommand{\ErrorTok}[1]{\textcolor[rgb]{0.64,0.00,0.00}{\textbf{#1}}}
\newcommand{\ExtensionTok}[1]{#1}
\newcommand{\FloatTok}[1]{\textcolor[rgb]{0.00,0.00,0.81}{#1}}
\newcommand{\FunctionTok}[1]{\textcolor[rgb]{0.13,0.29,0.53}{\textbf{#1}}}
\newcommand{\ImportTok}[1]{#1}
\newcommand{\InformationTok}[1]{\textcolor[rgb]{0.56,0.35,0.01}{\textbf{\textit{#1}}}}
\newcommand{\KeywordTok}[1]{\textcolor[rgb]{0.13,0.29,0.53}{\textbf{#1}}}
\newcommand{\NormalTok}[1]{#1}
\newcommand{\OperatorTok}[1]{\textcolor[rgb]{0.81,0.36,0.00}{\textbf{#1}}}
\newcommand{\OtherTok}[1]{\textcolor[rgb]{0.56,0.35,0.01}{#1}}
\newcommand{\PreprocessorTok}[1]{\textcolor[rgb]{0.56,0.35,0.01}{\textit{#1}}}
\newcommand{\RegionMarkerTok}[1]{#1}
\newcommand{\SpecialCharTok}[1]{\textcolor[rgb]{0.81,0.36,0.00}{\textbf{#1}}}
\newcommand{\SpecialStringTok}[1]{\textcolor[rgb]{0.31,0.60,0.02}{#1}}
\newcommand{\StringTok}[1]{\textcolor[rgb]{0.31,0.60,0.02}{#1}}
\newcommand{\VariableTok}[1]{\textcolor[rgb]{0.00,0.00,0.00}{#1}}
\newcommand{\VerbatimStringTok}[1]{\textcolor[rgb]{0.31,0.60,0.02}{#1}}
\newcommand{\WarningTok}[1]{\textcolor[rgb]{0.56,0.35,0.01}{\textbf{\textit{#1}}}}
\usepackage{longtable,booktabs,array}
\usepackage{calc} % for calculating minipage widths
\usepackage{caption}
% Make caption package work with longtable
\makeatletter
\def\fnum@table{\tablename~\thetable}
\makeatother
\usepackage{graphicx}
\makeatletter
\def\maxwidth{\ifdim\Gin@nat@width>\linewidth\linewidth\else\Gin@nat@width\fi}
\def\maxheight{\ifdim\Gin@nat@height>\textheight\textheight\else\Gin@nat@height\fi}
\makeatother
% Scale images if necessary, so that they will not overflow the page
% margins by default, and it is still possible to overwrite the defaults
% using explicit options in \includegraphics[width, height, ...]{}
\setkeys{Gin}{width=\maxwidth,height=\maxheight,keepaspectratio}
% Set default figure placement to htbp
\makeatletter
\def\fps@figure{htbp}
\makeatother
\setlength{\emergencystretch}{3em} % prevent overfull lines
\providecommand{\tightlist}{%
  \setlength{\itemsep}{0pt}\setlength{\parskip}{0pt}}
\setcounter{secnumdepth}{-\maxdimen} % remove section numbering
%% Custom LaTeX code to include in Beamer presentations from RMarkdown
%% Jonathan Whiteley, 2023
%% To be included in RMarkdown files:
%%    includes:
%%      in_header: "preamble-include.tex"
\setbeamertemplate{itemize item}[circle]
\setbeamertemplate{enumerate item}[circle]
%\useoutertheme{miniframes}    % for progress bars along the top of each frame
%\usepackage{layout}    % add `\layout` to the document to show page layout and parameter values
\usepackage{tikz}
\usetikzlibrary{arrows.meta,calc,tikzmark,fit, positioning}
% https://tex.stackexchange.com/questions/136143/tikz-animated-figure-in-beamer
% This won't hide parts of a drawing that have an opacity set in the picture or manually (it overrides the opacity here).
% see also the `overlay-beamer-styles` tikz library: https://tex.stackexchange.com/questions/639922/using-tikz-with-pause-command-in-rmarkdown-or-quarto-presentation
\tikzset{
  invisible/.style={opacity=0},
  visible on/.style={alt={#1{}{invisible}}},
  alt/.code args={<#1>#2#3}{%
    \alt<#1>{\pgfkeysalso{#2}}{\pgfkeysalso{#3}} % \pgfkeysalso doesn't change the path
  },
}
\AtBeginEnvironment{verbatim}{\small}    % reduce font size in R output
% https://stackoverflow.com/questions/35734525/reduce-space-between-code-chunks-and-code-output-in-rmarkdown-beamer-presentatio
\makeatletter
  \preto{\@verbatim}{\topsep=0pt \partopsep=0pt \itemsep=0pt}
\makeatother
% Set some lengths explicitly, for reproducibility across environments with different defaults
% Using fixed values (without rubber lengths) allows for more predictability in sizes,
%  which is useful for graphical overlays, though it may not be as flexible.
% See p. 145 of "The LaTeX Companion" (2e) for a diagram and explanation of some of these parameters.
\setlength{\partopsep}{0pt}
\setlength{\topsep}{0pt}
% My first instinct is to set \parsep to 0 (so paragraphs in the same item are closer together)
%  and set \itemsep to a rubber value; but the default is the opposite!
%  So all paragraphs are the same distance apart, regardless of item number...
% The \tightlist command sets \itemsep and \parskip to 0, but does not change \parsep.
\setlength{\parsep}{6pt plus2pt minus2pt}
\setlength{\itemsep}{0pt}
% For the most part, I don't actually want any space around the Shaded environments,
%  but it's the easiest way I found to reliably reserve space above the code chunks in columns underneath code output (verbatim).
\makeatletter
\@ifpackageloaded{framed}{
  \setlength{\OuterFrameSep}{4pt}  % = \topsep by default in `framed` package (but this is only applied at the start of the environment: the default value of \OuterFrameSep is otherwise \maxdimen!)
  \newlength\ShadedFrameSep 
  \ifdim\OuterFrameSep=\maxdimen 
    \ShadedFrameSep=\topsep \else 
    \ShadedFrameSep=\OuterFrameSep 
  \fi
}{}
\makeatother
%% Custom colours
\definecolor{darkred}{rgb}{0.8,0,0}
\definecolor{darkblue}{rgb}{0,0,0.6}
%% Custom macros
\newcommand{\ctop}{\vspace{\ShadedFrameSep}}  % for aligning text with the top of a code block in a table (0.2\baselinesep seemed like a good approximation, depending on other parametes; '\topsep+\parskip+\partopsep' seemed logical, but might be too much, maybe because it loses the rubber values)
\newcommand{\chup}{\vspace{-\ShadedFrameSep}}  % move code chunk up, to gobble whitespace above?
%\newcommand{\Rlogo}{\includegraphics[height=1em]{`r file.path(R.home("doc"), "html", "logo.jpg")`}}  %% needs to go through `knit()` to be valid.
\newcommand{\R}{\texttt{R}}
\newcommand{\highlight}[1]{\StringTok{#1}}
\newcommand{\important}[1]{\textcolor{darkred}{#1}}
\newcommand{\fade}[1]{\textcolor[rgb]{0.66,0.66,0.66}{#1}}
\newcommand{\annote}[1]{{\footnotesize #1}}
\newcommand{\name}[1]{\VariableTok{\texttt{#1}}}

\ifLuaTeX
  \usepackage{selnolig}  % disable illegal ligatures
\fi
\IfFileExists{bookmark.sty}{\usepackage{bookmark}}{\usepackage{hyperref}}
\IfFileExists{xurl.sty}{\usepackage{xurl}}{} % add URL line breaks if available
\urlstyle{same}
\hypersetup{
  pdftitle={A Short Introduction to Working With Data in R},
  pdfauthor={Jonathan Whiteley},
  colorlinks=true,
  linkcolor={darkblue},
  filecolor={Maroon},
  citecolor={Blue},
  urlcolor={darkblue},
  pdfcreator={LaTeX via pandoc}}

\title{A Short Introduction to Working With Data in R}
\author{Jonathan Whiteley}
\date{2023-09-05}

\begin{document}
\frame{\titlepage}

\begin{frame}[fragile]{Prerequisites}
\protect\hypertarget{prerequisites}{}
\begin{itemize}
\item
  Access to a copy of the
  \href{https://www.r-project.org}{\includegraphics[width=\textwidth,height=1em]{/Library/Frameworks/R.framework/Resources/doc/html/logo.jpg}}
  software

  \begin{itemize}
  \item
    i.e., a ``binary executable''
  \item
    Go to \href{https://www.r-project.org}{\emph{www.r-project.org}} to
    get a copy,\\
    or ask your system administrator.
  \end{itemize}
\item
  Tidyverse packages installed on the same system as \(\R\)

  \begin{itemize}
  \item
    Please run this command in \(\R\) \emph{before} the workshop:

\begin{Shaded}
\begin{Highlighting}[]
\FunctionTok{install.packages}\NormalTok{(}\StringTok{"tidyverse"}\NormalTok{)}
\end{Highlighting}
\end{Shaded}
  \end{itemize}
\item
  Knowledge of common mathematical operations: arithmetic, logarithms,
  etc.
\item
  Knowledge of basic \(\R\) concepts, such as \emph{variables},
  \emph{objects}, \emph{operators}, \emph{functions}, \emph{packages},
  etc.

  \begin{itemize}
  \tightlist
  \item
    This is covered in the first workshop: ``A Gentle Introduction to
    R''
  \end{itemize}
\end{itemize}
\end{frame}

\begin{frame}{Learning Objectives}
\protect\hypertarget{learning-objectives}{}
\begin{itemize}
\tightlist
\item
  Load tabular data into R
\item
  Explore data to check that it was loaded correctly
\item
  Export data from R to external files
\item
  Data frames
\item
  Clean data

  \begin{itemize}
  \tightlist
  \item
    Add \& change columns
  \item
    Edit values systematically
  \item
    Change data types
  \end{itemize}
\item
  Tidy data

  \begin{itemize}
  \tightlist
  \item
    Change the \emph{shape} of a data frame
  \end{itemize}
\item
  Re-use code, reproducible results, automated reports

  \begin{itemize}
  \tightlist
  \item
    Scripts
  \item
    R Markdown, R Notebooks
  \end{itemize}
\end{itemize}
\end{frame}

\hypertarget{welcome}{%
\section{Welcome}\label{welcome}}

\begin{frame}[fragile]{Pop Quiz}
\protect\hypertarget{pop-quiz}{}
\annote{\fade{We will review these \textit{at the end}, so you can see how much you have learned.}}

\begin{itemize}
\tightlist
\item
  If multiple packages have functions with the same name, how can you
  specify which one to use?
\item
  Does \(\R\) store data in memory or temporary files?
\item
  What is the limit to the size of objects and datasets that can be
  loaded into \(\R\)?
\item
  TRUE or FALSE: \(\R\) has rules and conventions for naming functions
\item
  TRUE or FALSE: if you use one package from the \texttt{tidyverse}, you
  have to use all of them.
\end{itemize}

~

\begin{block}{Answer in the chat:}
\protect\hypertarget{answer-in-the-chat}{}
What is your favourite emoji? Why do you like to use it so much?
\end{block}
\end{frame}

\begin{frame}{Introductions}
\protect\hypertarget{introductions}{}
\begin{itemize}
\tightlist
\item
  Name
\item
  Pronouns
\item
  Job title, role
\item
  \emph{optional}: a favourite childhood treat or candy?
\end{itemize}

\begin{itemize}
\tightlist
\item
  What are you hoping to learn most in today's workshop?
\end{itemize}
\end{frame}

\begin{frame}{Disclaimer}
\protect\hypertarget{disclaimer}{}
\begin{itemize}
\item
  There is often more than one way to achieve a desired result in \(\R\)
\item
  Some are faster in certain situations
\item
  Some require less code, or are easier to write as code
\item
  Some are more portable (work on multiple systems)
\item
  But there is rarely as single `best way'.
\end{itemize}

This workshop focuses on a coherent approach, that can be learned more
easily and extended as needed to tackle bigger problems.

Feel free to take what you learn here and experiment, or explore
alternatives. Find what works for \emph{you}.
\end{frame}

\hypertarget{loading-data-into-r}{%
\section{\texorpdfstring{Loading data into
\(\R\)}{Loading data into \textbackslash R}}\label{loading-data-into-r}}

\begin{frame}{}
\protect\hypertarget{section}{}
\end{frame}

\hypertarget{exploring-your-data}{%
\section{Exploring your data}\label{exploring-your-data}}

\begin{frame}{Data frames}
\protect\hypertarget{data-frames}{}
\end{frame}

\hypertarget{saving-data-outside-r}{%
\section{\texorpdfstring{Saving data outside
\(\R\)}{Saving data outside \textbackslash R}}\label{saving-data-outside-r}}

\begin{frame}{Saving data outside \(\R\)}
\end{frame}

\hypertarget{re-using-your-code-scripts-and-other-files}{%
\section{Re-using your code: scripts and other
files}\label{re-using-your-code-scripts-and-other-files}}

\begin{frame}{Re-using your code: scripts and other files}
\end{frame}

\hypertarget{the-tidyverse-collection-of-packages}{%
\section{\texorpdfstring{The \texttt{tidyverse} collection of
packages}{The tidyverse collection of packages}}\label{the-tidyverse-collection-of-packages}}

\begin{frame}[fragile]{The \texttt{tidyverse}}
\protect\hypertarget{the-tidyverse}{}
\begin{Shaded}
\begin{Highlighting}[]
\FunctionTok{install.packages}\NormalTok{(}\StringTok{"tidyverse"}\NormalTok{)}
\FunctionTok{help}\NormalTok{(}\AttributeTok{package=}\StringTok{"tidyverse"}\NormalTok{)}
\end{Highlighting}
\end{Shaded}

\begin{itemize}
\item
  The \href{https://www.tidyverse.org/}{\texttt{tidyverse}} is an
  ``opinionated'' \href{https://www.tidyverse.org/packages/}{collection
  of packages} that are designed to work together.
\item
  All packages share an underlying design philosophy, grammar, and data
  structures.

  \begin{itemize}
  \tightlist
  \item
    \emph{Unlike base \(\R\)}
  \item
    Shared naming conventions (e.g., `\texttt{\_}' instead of
    `\texttt{.}' in function names)
  \item
    Emphasis on functions that do one thing well
  \item
    Designed to be combined together to achieve complex operations
  \end{itemize}
\item
  \texttt{tidyverse} is under active development.

  \begin{itemize}
  \tightlist
  \item
    New functions and features sometimes replace or supersede old ones.
  \item
    No guarantee that functions will continue to work the same way in
    future versions.
  \end{itemize}
\end{itemize}

\note{These characteristics means tidyverse packages may not be ideal in
a production environment. Nevertheless, the package designers are pretty
good about replacing old functions with new ones, to avoid disruptive
changes, and keeping older functions around, though with only minimal
support. I have code from 10+ years ago that will not run in current
versions of R because of the number of changes to dplyr and related
packages over the years.

But I still like dplyr and the tidyverse, because they are coherent with
each other. Although they can be complex, and take advantage of some
arcane aspects of R, they do make it easier to translate ideas into code
--- once you understand the grammar. The consistency in naming and
argument syntax, however, is also hugely appealing in terms of fewer new
things to learn while expanding your toolbox.

Write code that works and that you understand: then take time to revise
and optimize, based on your needs and capacity.}
\end{frame}

\begin{frame}[fragile]{Core \texttt{tidyverse} packages}
\protect\hypertarget{core-tidyverse-packages}{}
Today, we will focus on a few of the core \texttt{tidyverse} packages
for loading, cleaning, and manipulating data:

\begin{itemize}
\tightlist
\item
  \href{https://readr.tidyverse.org/}{readr},
  \href{https://readxl.tidyverse.org/}{readxl} for \textbf{loading} data
\item
  \href{https://dplyr.tidyverse.org/}{dplyr} for \textbf{manipulating}
  data (values)
\item
  \href{https://tidyr.tidyverse.org/}{tidyr} for \textbf{rearranging}
  data
\item
  \href{https://stringr.tidyverse.org/}{stringr} for working with
  \textbf{strings}
\end{itemize}
\end{frame}

\begin{frame}[fragile]{\texttt{dplyr}: grammar of data manipulation}
\protect\hypertarget{dplyr-grammar-of-data-manipulation}{}
\begin{itemize}
\item
  \texttt{dplyr} provides many functions, within a coherent framework or
  \emph{grammar}
\item
  They are intended to help you focus on \emph{what} you want to do, and
  translate your thoughts into code.
\item
  High-level functions have active names and called ``\textbf{verbs}''
  --- they describe what they do.
\item
  \texttt{dplyr} and \texttt{tidyr} provide many ``\textbf{helper
  functions}'' that work \emph{inside} verbs and other functions to make
  many tasks easier to translate into code.

  \begin{itemize}
  \tightlist
  \item
    These functions may not work on their own, outside of \texttt{dplyr}
    verbs and \texttt{tidyr} functions.
  \end{itemize}
\end{itemize}
\end{frame}

\begin{frame}[fragile]{\texttt{dplyr} verbs}
\protect\hypertarget{dplyr-verbs}{}
Verbs can be grouped based on the component of the dataset that they
work with\footnote<.->{\url{https://dplyr.tidyverse.org/articles/dplyr.html\#single-table-verbs}}:

\begin{itemize}
\tightlist
\item
  Rows:

  \begin{itemize}
  \tightlist
  \item
    \VERB|\FunctionTok{filter}\NormalTok{()}| chooses rows based on
    column values.
  \item
    \VERB|\FunctionTok{slice}\NormalTok{()}| chooses rows based on
    location.
  \item
    \VERB|\FunctionTok{arrange}\NormalTok{()}| changes the order of the
    rows.
  \end{itemize}
\item
  Columns:

  \begin{itemize}
  \tightlist
  \item
    \VERB|\FunctionTok{select}\NormalTok{()}| changes whether or not a
    column is included.
  \item
    \VERB|\FunctionTok{rename}\NormalTok{()}| changes the name of
    columns.
  \item
    \VERB|\FunctionTok{mutate}\NormalTok{()}| changes the \emph{values}
    of columns and creates new columns.
  \item
    \VERB|\FunctionTok{relocate}\NormalTok{()}| changes the order of the
    columns.
  \end{itemize}
\item
  Groups of rows:

  \begin{itemize}
  \tightlist
  \item
    \VERB|\FunctionTok{group\_by}\NormalTok{()}| defines groups of rows.
  \item
    \VERB|\FunctionTok{summarise}\NormalTok{()}| collapses a group into
    a single row.
  \end{itemize}
\end{itemize}

\note{}
\end{frame}

\begin{frame}[fragile,shrink]{A `pipe' operator}
\protect\hypertarget{a-pipe-operator}{}
\begin{columns}[c,onlytextwidth]
\begin{column}{0.38\textwidth}
\begin{figure}
\centering
\includegraphics{images/MagrittePipe.jpg}
\caption{\href{https://en.wikipedia.org/wiki/The_Treachery_of_Images}{``La
Trahison des Images'' (``The Treachery of Images'')} or ``Ceci~n'est pas
une pipe'' (``This~is not a pipe'') by René Magritte. }
\end{figure}

\centering

\href{https://magrittr.tidyverse.org/}{\includegraphics[width=\textwidth,height=0.25\textheight]{images/magrittr_logo.png}}
\end{column}

\begin{column}{0.58\textwidth}
\begin{itemize}
\item
  The \href{https://magrittr.tidyverse.org/}{\texttt{magrittr}}
  \href{https://cran.r-project.org/package=magrittr}{package} (included
  with
  \href{https://www.tidyverse.org/packages/\#program}{\texttt{tidyverse}})
  provides a ``forward-pipe operator'':

\begin{Shaded}
\begin{Highlighting}[]
\SpecialCharTok{\%\textgreater{}\%}    \CommentTok{\# ?magrittr::\textasciigrave{}\%\textgreater{}\%\textasciigrave{}}
\end{Highlighting}
\end{Shaded}
\item
  The \texttt{magrittr} package is automatically loaded when loading
  most \texttt{tidyverse} packages (e.g., \texttt{tidyr},
  \texttt{dplyr}, \texttt{ggplot2}), as these packages all use this
  operator extensively.

  \begin{itemize}
  \tightlist
  \item
    It is often unnecessary to load \texttt{magrittr} separately, unless
    you are \textbf{not} using these other packages.
  \end{itemize}
\end{itemize}
\end{column}
\end{columns}

\note{\texttt{magrittr} logo downloaded from:
\url{https://github.com/tidyverse/magrittr/blob/main/man/figures/logo.png}}
\end{frame}

\begin{frame}[fragile]{\texttt{magrittr}`s 'forward-pipe' operator}
\protect\hypertarget{magrittrs-forward-pipe-operator}{}
\begin{itemize}
\tightlist
\item
  \VERB|\SpecialCharTok{\%\textgreater{}\%}| allows you to pass results
  from an expression on the left-hand side (LHS) as an argument (usually
  the first) to a \emph{function call} on the right-hand side (RHS).
\end{itemize}

\begin{longtable}[]{@{}
  >{\raggedright\arraybackslash}p{(\columnwidth - 2\tabcolsep) * \real{0.4583}}
  >{\raggedright\arraybackslash}p{(\columnwidth - 2\tabcolsep) * \real{0.4583}}@{}}
\toprule\noalign{}
\begin{minipage}[b]{\linewidth}\raggedright
This expression \ldots{}
\end{minipage} & \begin{minipage}[b]{\linewidth}\raggedright
is equivalent to:
\end{minipage} \\
\midrule\noalign{}
\endhead
\begin{minipage}[t]{\linewidth}\raggedright
\begin{Shaded}
\begin{Highlighting}[]
\NormalTok{x }\SpecialCharTok{\%\textgreater{}\%} \FunctionTok{f}\NormalTok{()}
\end{Highlighting}
\end{Shaded}
\end{minipage} & \begin{minipage}[t]{\linewidth}\raggedright
\begin{Shaded}
\begin{Highlighting}[]
\FunctionTok{f}\NormalTok{(x)}
\end{Highlighting}
\end{Shaded}
\end{minipage} \\
\begin{minipage}[t]{\linewidth}\raggedright
\begin{Shaded}
\begin{Highlighting}[]
\NormalTok{x }\SpecialCharTok{\%\textgreater{}\%} \FunctionTok{f}\NormalTok{(y)}
\end{Highlighting}
\end{Shaded}
\end{minipage} & \begin{minipage}[t]{\linewidth}\raggedright
\begin{Shaded}
\begin{Highlighting}[]
\FunctionTok{f}\NormalTok{(x, y)}
\end{Highlighting}
\end{Shaded}
\end{minipage} \\
\begin{minipage}[t]{\linewidth}\raggedright
\begin{Shaded}
\begin{Highlighting}[]
\NormalTok{x }\SpecialCharTok{\%\textgreater{}\%} \FunctionTok{f}\NormalTok{(y, }\AttributeTok{z =}\NormalTok{ .)}
\end{Highlighting}
\end{Shaded}
\end{minipage} & \begin{minipage}[t]{\linewidth}\raggedright
\begin{Shaded}
\begin{Highlighting}[]
\FunctionTok{f}\NormalTok{(y, }\AttributeTok{z =}\NormalTok{ x)}
\end{Highlighting}
\end{Shaded}
\end{minipage} \\
\begin{minipage}[t]{\linewidth}\raggedright
\begin{Shaded}
\begin{Highlighting}[]
\NormalTok{x }\SpecialCharTok{\%\textgreater{}\%}\NormalTok{ f }\SpecialCharTok{\%\textgreater{}\%}\NormalTok{ g }\SpecialCharTok{\%\textgreater{}\%}\NormalTok{ h}
\end{Highlighting}
\end{Shaded}
\end{minipage} & \begin{minipage}[t]{\linewidth}\raggedright
\begin{Shaded}
\begin{Highlighting}[]
\FunctionTok{h}\NormalTok{(}\FunctionTok{g}\NormalTok{(}\FunctionTok{f}\NormalTok{(x)))}
\end{Highlighting}
\end{Shaded}
\end{minipage} \\
\bottomrule\noalign{}
\end{longtable}

\begin{itemize}
\tightlist
\item
  This can make code easier to read, as expressions are written and
  evaluated from \emph{left to right}, rather than from \emph{inside to
  outside} nested parentheses.
\end{itemize}
\end{frame}

\begin{frame}[fragile]{\(\R\) now has a `native' pipe operator}
\protect\hypertarget{r-now-has-a-native-pipe-operator}{}
\begin{itemize}
\item
  A pipe operator was introduced in base \(\R\) in v4.1 (May
  2021)\footnote<.->{\url{https://cran.r-project.org/bin/windows/base/old/4.1.0/NEWS.R-4.1.0.html}}:

\begin{Shaded}
\begin{Highlighting}[]
\SpecialCharTok{|\textgreater{}}    \CommentTok{\# ?pipeOp}
\end{Highlighting}
\end{Shaded}
\item
  It was inspired by the ``forward pipe operator'' introduced by
  \texttt{magrittr}, but is more streamlined. See these links for
  details:

  \begin{itemize}
  \tightlist
  \item
    \href{https://www.tidyverse.org/blog/2023/04/base-vs-magrittr-pipe/}{Differences
    between the base R and magrittr pipes}
  \item
    ``\href{https://towardsdatascience.com/understanding-the-native-r-pipe-98dea6d8b61b}{Understanding
    the native R pipe \textbar\textgreater{}}''
  \end{itemize}
\item
  Because it is so new, most code examples online still use
  `\VERB|\SpecialCharTok{\%\textgreater{}\%}|' from \texttt{magrittr}.
\item
  But `\VERB|\SpecialCharTok{\VerbBar{}\textgreater{}}|' is always
  available \emph{in \(\R\) \textgreater= v4.1}, without having to load
  additional packages.
\item
  This document will use `\VERB|\SpecialCharTok{\%\textgreater{}\%}|' in
  the examples, for consistency and because many \texttt{tidyverse}
  functions were designed to work with it.
\end{itemize}
\end{frame}

\begin{frame}{Pipes: exercise}
\protect\hypertarget{pipes-exercise}{}
\end{frame}

\hypertarget{clean-data}{%
\section{Clean data}\label{clean-data}}

\begin{frame}{Clean data}
\end{frame}

\hypertarget{tidy-data}{%
\section{Tidy data}\label{tidy-data}}

\begin{frame}[fragile]{Tidy datasets}
\protect\hypertarget{tidy-datasets}{}
\begin{quote}
\begin{quote}
Happy families are all alike; every unhappy family is unhappy in its own
way
\end{quote}

--- Leo Tolstoy\\
\strut \\
Like families, tidy datasets are all alike but every messy dataset is
messy in its own way.

--- Hadley Wickham (doi:
\href{https://doi.org/10.18637/jss.v059.i10}{10.18637/jss.v059.i10})
\end{quote}

\begin{itemize}
\item
  Tidy datasets provide a standardized way to link the \emph{structure}
  of a dataset (its physical layout) with its \emph{semantics} (its
  meaning).

  \begin{itemize}
  \tightlist
  \item
    \href{https://cran.r-project.org/web/packages/tidyr/vignettes/tidy-data.html}{\texttt{tidyr}
    vignette}
  \end{itemize}
\end{itemize}
\end{frame}

\hypertarget{review}{%
\section{Review}\label{review}}

\begin{frame}{Exercise}
\protect\hypertarget{exercise}{}
\end{frame}

\begin{frame}[fragile]{\protect\hyperlink{pop-quiz}{Quiz} Review}
\protect\hypertarget{quiz-review}{}
\note{\texttt{\#\#\ Answers\ (for\ discussion)}}
\end{frame}

\hypertarget{backmatter}{%
\section{Backmatter}\label{backmatter}}

\begin{frame}[fragile]{Other packages to look at}
\protect\hypertarget{other-packages-to-look-at}{}
\begin{itemize}
\tightlist
\item
  \href{https://rdatatable.gitlab.io/data.table/}{\texttt{data.table}}:
  a high-performance version of \texttt{data.frame} with few
  dependencies.
\end{itemize}

Other packages in the \texttt{tidyverse}:

\begin{itemize}
\item
  \href{https://lubridate.tidyverse.org/}{\texttt{lubridate}} and
  \href{https://hms.tidyverse.org/}{\texttt{hms}}: for dates \& time
  values.
\item
  \href{https://purrr.tidyverse.org/}{\texttt{purrr}}: functional
  programming (FP) tools for working with functions and vectors.

  \begin{itemize}
  \tightlist
  \item
    Replace \texttt{for} loops with code that is more efficient and
    easier to read.
  \end{itemize}
\end{itemize}
\end{frame}

\begin{frame}[fragile,shrink]{Writing to Microsoft
Excel\textsuperscript{TM} files}
\protect\hypertarget{writing-to-microsoft-exceltm-files}{}
Packages that can write to Excel files:

\begin{itemize}
\tightlist
\item
  \href{https://github.com/colearendt/xlsx}{\texttt{xlsx}}: read, write,
  format Excel~2007 (\texttt{.xlsx}) and Excel~97/2000/XP/2003
  (\texttt{.xls}) files.

  \begin{itemize}
  \tightlist
  \item
    Depends on Java and the \texttt{rJava} package
  \end{itemize}
\item
  \href{https://github.com/miraisolutions/xlconnect}{\texttt{XLConnect}}:
  comprehensive and cross-platform R package for manipulating Microsoft
  Excel files (\texttt{.xlsx} \& \texttt{.xls}) from within R.

  \begin{itemize}
  \tightlist
  \item
    Requires a Java Runtime Environment (JRE)
  \end{itemize}
\item
  \href{https://ycphs.github.io/openxlsx/index.html}{\texttt{openxlsx}}:
  simplified creation of Excel \texttt{.xlsx} files (\textbf{not}
  \texttt{.xls}).

  \begin{itemize}
  \tightlist
  \item
    \emph{No dependency} on Java
  \end{itemize}
\item
  \href{https://docs.ropensci.org/writexl/}{\texttt{writexl}}: portable,
  light-weight data frame to \textbf{xlsx} exporter.

  \begin{itemize}
  \tightlist
  \item
    No Java or Excel required
  \end{itemize}
\end{itemize}

\begin{block}{!}
\protect\hypertarget{section-1}{}
I recommend \emph{avoiding} exporting data to Excel files if possible.
\texttt{csv} files are easier to read to \& write from, and can be read
by a wider variety of software (they are more portable).\\
Automated reports can be produced with R Markdown and output to a
variety of more portable formats (pdf, HTML, etc.) instead.

\note{As seen here, there are several packages for working with Excel
files, with different advantages and disadvantages: you may have to do
some testing to find the best fit for your situation.

I personally prefer the \texttt{openxlsx} package, because in my
experience, the dependency on Java can be difficult to get working and
unreliable. But I have no experience with \texttt{writexl}, which is
newer and possibly faster than the others.\\
The main trade-off is the openxlsx cannot write to \texttt{.xls} files
(Excel 97/2000/XP/2003 format) and has fewer formatting options than
other packages --- but if you need that much formatting, why not produce
a formatted report with R Markdown?}
\end{block}
\end{frame}

\begin{frame}{References}
\protect\hypertarget{references}{}
Cheatsheets:

\begin{itemize}
\tightlist
\item
  \href{https://rstudio.github.io/cheatsheets/html/data-import.html}{readr/readxl}
\item
  \href{https://rstudio.github.io/cheatsheets/html/data-transformation.html}{Data
  transformation with dplyr}
\item
  \href{https://rstudio.github.io/cheatsheets/html/tidyr.html}{Data
  tidying with tidyr}
\end{itemize}

\note{\url{https://cran.rapporter.net/doc/contrib/de_Jonge+van_der_Loo-Introduction_to_data_cleaning_with_R.pdf}
\url{https://r4ds.hadley.nz/data-transform.html}
(\url{https://r4ds.had.co.nz/transform.html})
\url{https://github.com/rstudio-education/r4ds-instructors}
\url{https://dplyr.tidyverse.org/articles/dplyr.html\#single-table-verbs}
\url{https://dplyr.tidyverse.org/articles/two-table.html}
\url{https://education.rstudio.com/teach/materials/}
\url{https://github.com/rstudio-education/remaster-the-tidyverse/blob/master/README.md}
\url{https://datasciencebox.org/}
\url{https://rladiessydney.org/courses/ryouwithme/02-cleanitup-0/}
\url{https://posit.cloud/learn/primers}

CANSIM data: \url{https://mountainmath.ca/canssi/\#1}
\url{https://mountainmath.github.io/canadian_data/}

Working directory \& scripts:
\url{https://stackoverflow.com/questions/3452086/getting-path-of-an-r-script}
\url{https://stackoverflow.com/questions/47044068/get-the-path-of-current-script}
\url{https://cran.r-project.org/web/packages/this.path/this.path.pdf}

Other related tutorials:
\url{https://posit.co/resources/videos/a-gentle-introduction-to-tidy-statistics-in-r/}}
\end{frame}

\end{document}
